\begin{hcarentry}[section]{yi}
\label{yi}
\report{Jean-Philippe Bernardy}
\status{active development}
\makeheader

Yi is a project to write a Haskell-extensible editor. Yi is structured
around a purely functional editor core, such that most components of
the editor can be overridden by the user, using configuration files
written in Haskell.

Yi has been converted to the Cabal build system, which makes it easier to
build and experiment with.

Yi features:
\begin{compactitem}
\item Keybindings for emacs and vim, written as extensible parsers;
\item Vty and Gtk2Hs frontends;
\item Syntax highlighting for haskell and other languages;
\item XMonad-style static configuration;
\item Support of Linux, MacOS and Windows platforms.
\end{compactitem}

We are currently working on the following fronts:
\begin{compactitem}
\item Integration with Cabal and GHC API;
\item Syntax-aware support of Haskell;
\item Pango, cocoa frontends
\end{compactitem}

\FurtherReading
\begin{compactitem}
\item Documentation can be found at:

 \url{http://haskell.org/haskellwiki/Yi}

\item The source repository is available:

 \texttt{darcs get}

 \text{\url{http://code.haskell.org/yi/}}
\end{compactitem}
\end{hcarentry}
