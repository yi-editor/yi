\begin{hcarentry}[section]{yi}
\label{yi}
\report{Jean-Philippe Bernardy}
\participants{Don Stewart}
\status{active development}
\makeheader

Yi is an editor written in Haskell and extensible in Haskell. We leverage the
expressiveness of Haskell to provide an editor which is powerful and easy to
extend.

We have recently made Yi much more accessible. On unix systems, it can be
deployed using a single \textbf{cabal install} command. We also polished the
user interface and behaviour to facilitate the transition from emacs or vim.

Yi features:
\begin{compactitem}
\item Keybindings for emacs and vim, written as extensible parsers;
\item Vty, Gtk2Hs frontends;
\item Syntax highlighting for haskell and other languages;
\item XMonad-style static configuration;
\item Support of Linux, MacOS and Windows platforms.
\item Special support for haskell: layout-aware edition, paren-matching, GHCi interface, Cabal build interface, ...
\end{compactitem}
We are currently working on the following fronts:
\begin{compactitem}
\item Integration with Cabal and GHC API;
\item Pango and Cocoa frontends;
\item CUA keybindings
\end{compactitem}

\FurtherReading
\begin{compactitem}
\item More information can be found at:
 \url{http://haskell.org/haskellwiki/Yi}

\item The source repository is available:
 \texttt{darcs get}
 \text{\url{http://code.haskell.org/yi/}}
\end{compactitem}
\end{hcarentry}
