\documentclass[utf8x,compress]{beamer}
\usepackage[utf8x]{inputenc}
% \DeclareUnicodeCharacter{981}{\phi}
% \usepackage{beamerthemesplit}
%\usepackage{aop-text}
%\usepackage{aop-math}

\usetheme{Malmoe}

\title{Yi}
\subtitle{An Editor in Haskell for Haskell}
\author{Jean-Philippe Bernardy}
% \email           {bernardy@chalmers.se}
\institute {Chalmers University of Technology
            % and University of Gothenburg
          }
\date {
      Haskell Symposium 2008
      \\Thursday, 25th September
      }

\begin{document}

\frame{\titlepage}

\frame{
  \frametitle{The Project}
  \begin{itemize}
    \item Haskell for text editor
    \item Text editor for Haskell
    \item Started in 2004
    \item $>$ 2500 patches
    \item 47 contributors
    \item about 15kloc
  \end{itemize}
}


\frame{
  \frametitle{Features}
      \begin{itemize}
        \item Usable editor
        \begin{itemize}
          \item Buffers, Copy/Paste, Search/Replace, Undo, ...
          \item Windows, Tabs, Syntax highlighting, ...
        \end{itemize}
        \item Keymaps: vim, emacs, (cua), ...
        \item Frontends: {\textbf vty}, gtk, (cocoa), ...
        \item Modes: Haskell, \LaTeX, perl, python, ...
        \item Configurable in Haskell
        \itemize{
          \item Yi is a library (as XMonad)
          \item Users ``configure'' Yi by combining building blocks 
          \item no patching (emacs)
        }
      \end{itemize}
}

\frame{
  \frametitle{Architecture}
    \begin{itemize}
      \item Haskell + Hackage
      \item Abstracted UIs
      \item Stack of monads
      \begin{itemize}
         \item IO
         \item Editor
         \item Buffer
      \end{itemize}
      \item Keymap = Input $\leadsto$ Action 
      \begin{itemize}
        \item parsers (combinators) of input
      \end{itemize}
      \item Language support
      \begin{itemize}
        \item Lexical analysis
        \item Syntax analysis (special parser combinators)
        \item AST-based highlighting
        \item AST available in mode-dependent functions
      \end{itemize}
    \end{itemize}
}


\frame{
  \frametitle{Haskell Support}
      \begin{itemize}
        \item Paren matching
        \begin{itemize}
          \item Layout aware
        \end{itemize}
        \item Auto indent
        \item Layout-preservation
        \item Ghci
        \item Cabal build
        \item (GHC API)
      \end{itemize}
}

\frame{
  \frametitle{Demo}
      \begin{itemize}
        \item Haskell Support
        \item Configuration
        \item Latex
      \end{itemize}
}



\end{document}