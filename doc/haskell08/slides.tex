\documentclass[utf8x]{beamer}
\usepackage[utf8x]{inputenc}
% \DeclareUnicodeCharacter{981}{\phi}
\usepackage{beamerthemesplit}
%\usepackage{aop-text}
%\usepackage{aop-math}

\title{Yi}
\subtitle{An Editor in Haskell for Haskell}
\author{Jean-Philippe Bernardy}
% \email           {bernardy@chalmers.se}
\institute {Chalmers University of Technology
            % and University of Gothenburg
          }
\date {
      Haskell Symposium 2008
      \\Thursday, 25th September
      }

\begin{document}

\frame{\titlepage}

\frame{\tableofcontents}

\section{Capabilities}
\frame{
  \frametitle{Capabilities}
      \begin{itemize}
        \item Purely functional core + IO shell
        \item Keymap as parsers: vim, emacs, cua, ...
        \item Frontends: vty, gtk, cocoa, ...
        \item Syntax-aware
      \end{itemize}
}

\frame{
  \frametitle{Haskell Support}
      \begin{itemize}
        \item Paren matching
        \item Layout rule
        \item Cabal build
      \end{itemize}
}

\frame{
  \frametitle{Demo}
}

\section{Configuration}

\frame{
  \itemize{
    \item Yi is a library (as XMonad)
    \item Users "configure" Yi by combining the building blocks.
    \item Example!
  }
}

\end{document}