RUNTIME

Preferences are stored in ~/.hemacs/Config.hs. Look in plugins/ for
example Config.hs. If this directory doesn't exist, hemacs uses
HEmacs/ConfigAPI.hs for the defaults, this is also what is used in
hemacs-static.

To get at hemacs internals from your config script, either import
HEmacs.HEmacs, which brings most of the commonly used modules into
scope, or explicitly import something from HEmacs/*. You can read the
haddock-generated docs from some (preliminary) api details (make
html).

How do things load in the dynamic loader.
        
           Boot.hs
          /      \
      Config.hs HEmacs.o
                  |
                HShemacs.o (-package hemacs)

once it is all loaded, Boot.main jumps to HEmacs.dynamic\_main

BUILD SYSTEM

Extra args can be passed at runtime as: make HC\_OPTS=-ddump-minimal-imports

There is support for building mulitple "ways", i.e. prof, but it is
untested.

DEVELOPMENT

Coding guidelines:

Please make sure you code compiles with -Wall, with no warnings. Use
Haddock markup to document any functions that are exported -- user's
may want to use this code at some point. We use cpp to pass
platform-specific options into files. Please provide explicit import
lists where reasonable, and keep imports in some sorted order.
Please don't introduce posix dependencies unless you absolutely have
to.

You can actually run this stuff in ghci! To just run hemacs-core (the gui etc,
without worrying about booting the thing):

\begin{verbatim}
        ghci -package-conf hemacs.conf -package hemacs
        manzano$ ghci -package-conf hemacs.conf -package hemacs
           ___         ___ _
          / _ \ /\  /\/ __(_)
         / /_\// /_/ / /  | |      GHC Interactive, version 6.3, for Haskell 98.
        / /_\\/ __  / /___| |      http://www.haskell.org/ghc/
        \____/\/ /_/\____/|_|      Type :? for help.

        Loading package base ... linking ... done.
        Loading package haskell98 ... linking ... done.
        Loading package mtl ... linking ... done.
        Loading package lang ... linking ... done.
        Loading package unix ... linking ... done.
        Loading package posix ... linking ... done.
        Loading package hemacs ... linking ... done.
        Prelude> :l Main
        Prelude Main> main
\end{verbatim}

To run the whole thing is a bit harder, and requires faking the
package lib flags in hemacs.conf (leave off the last 2 args if you're
on linux). And if we haven't installed it yet (you want to run
in-tree), then you have to set the command line args. Also, to do this
trick you need ghc-6.2.2 or greater, which contain appropriate rts
patches for running dynamic loaders dynamically.

\begin{verbatim}
        manzano$ ghci -package plugins -liconv -L/usr/local/lib
           ___         ___ _
          / _ \ /\  /\/ __(_)
         / /_\// /_/ / /  | |      GHC Interactive, version 6.3, for Haskell 98.
        / /_\\/ __  / /___| |      http://www.haskell.org/ghc/
        \____/\/ /_/\____/|_|      Type :? for help.

        Loading package base ... linking ... done.
        Loading package altdata ... linking ... done.
        Loading package haskell98 ... linking ... done.
        Loading package hi ... linking ... done.
        Loading package unix ... linking ... done.
        Loading package haskell-src ... linking ... done.
        Loading package mtl ... linking ... done.
        Loading package lang ... linking ... done.
        Loading package posix ... linking ... done.
        Loading package plugins ... linking ... done.
        Loading object (dynamic) iconv ... done
        final link ... done
	Prelude> :set -cpp -fglasgow-exts
	Prelude> :set -DLIBDIR="/usr/local/lib/hemacs"
	Prelude> :set args -B/home/dons/src/projects/hemacs
	Prelude> :l Boot.hs
	*Boot>
        Prelude Boot> main
        Starting up dynamic Haskell ... jumping over the edge ... done.
\end{verbatim}

For example, to modify \code{\$(TOP)/HEmacs.hs} in ghci, you would:

\begin{verbatim}
        manzano$ ghci -package-conf hemacs.conf -package hemacs -cpp -fglasgow-exts
        Prelude> :l HEmacs.hs
        *HEmacs> :t dynamic_main
        dynamic_main :: Maybe ConfigData -> IO ()
        *HEmacs> :reload
        Compiling HEmacs           ( HEmacs.hs, interpreted )
\end{verbatim}

\begin{thebibliography}[50]
\bibitem Emacs, The extensible, customizable ... Stallman.
\end{thebibliography}
